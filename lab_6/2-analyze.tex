\chapter{Аналитический раздел}
В этом разделе будут рассмотрены задача о коммивояжёре и муравьиный алгоритм для её решения.

\section{Задача о коммивояжёре}
Задача коммивояжёра — одна из самых известных задач комбинаторной оптимизации, заключающаяся в поиске самого выгодного маршрута, 
проходящего через все указанные города по одному разу с последующим возвратом в исходный город.
Маршрут должен проходить через каждый город только один раз.\cite{commi}

Задача коммивояжёра относится к числу трансвычислительных: уже при относительно небольшом числе городов (66 и более) она не может быть решена методом перебора вариантов никакими теоретически мыслимыми компьютерами за время, меньшее нескольких миллиардов лет.

\section{Муравьиный алгоритм}
Идея муравьиного алгоритма -- моделирование поведения муравьев, связанное с их способностью быстро находить кратчайшие пути и адаптироваться к изменяющимся условиям, т.е. искать новые кратчайшие пути \cite{ant1}. 
При своём движении муравей метит свой путь феромоном, и эта информация используется прочими для выбора пути. 
Таким образом, более короткие пути будут сильнее обогащаться феромоном и станут более предпочтительны для всей колонии. 
С помощью моделирования испарение феромона, т.е. отрицательной обратной связи, гарантируется, что найденное локально оптимальное решение не будет единственным -- будут предприняты попытки поиска других путей.

Опишем правила поведения муравья применительно к решению задачи коммивояжера \cite{shtovba}:
\begin{itemize}
	\item муравьи имеют 'память' - запоминают уже посещенные города, чтобы избегать повторений: обозначим через $J_{i,k}$ список городов, которые необходимо пройти муравью $k$, находящемуся в городе $i$;
	\item муравьи обладают 'зрением' - видимость есть эвристическое желание посетить город $j$, если муравей находится в городе $i$. 
	
	Уместно считать видимость обратно пропорциональной расстоянию между соответствующими городами $\eta_{i,j} = 1/D_{i,j}$;
	\item муравьи обладают 'обонянием' -- могут улавливать след феромона, подтверждающий желание посетить город $j$ из города $i$ на основании опыта других муравьев. 

	Обозначим количество феромона на ребре $(i,j)$ в момент времени $t$ через $\tau_{i,j}(t)$. 
\end{itemize}

Вероятность перехода из вершины $i$ в вершину $j$ определяется по формуле 1.1 

\begin{equation}
p_{i,j}={\frac {(\tau_{i,j}^{\alpha })(\eta_{i,j}^{\beta })}{\sum (\tau_{i,j}^{\alpha})(\eta_{i,j}^{\beta })}}
\end{equation}

где  $ \tau_{i,j} - $ количество феромонов на ребре i до j;
$\eta_{i,j} - $ эвристическое расстояние от i до j;
$\alpha - $ параметр влияния феромона;
$\beta - $ параметр влияния расстояния.

Пройдя ребро $(i,j)$ , муравей откладывает на нём некоторое количество феромона, которое должно быть связано с оптимальностью сделанного выбора. 

Пусть $T _{k} (t)$ есть маршрут, пройденный муравьём $k$ к моменту времени $t$ , $L _{k} (t)$ - длина этого маршрута, а $Q$ - параметр, имеющий значение порядка длины оптимального пути.
Тогда откладываемое количество феромона может быть задано по формуле 1.2:

\begin{equation}
{\displaystyle \Delta \tau _{i,j}^k={\begin{cases}Q/L_{k}& {\mbox{Если k-ый муравей прошел по ребру ij;}}\\0&{\mbox{Иначе}}\end{cases}}}
\end{equation}

где Q - количество феромона, переносимого муравьём;


Тогда:

\begin{equation}
\Delta \tau _{i,j}= \tau _{i,j}^0 + \tau _{i,j}^1 + ... + \tau _{i,j}^k
\end{equation}

\section{Вывод}
Была рассмотрена задача о коммивояжёра и проанализировано применение муравьиного алгоритма для решения этой задачи.