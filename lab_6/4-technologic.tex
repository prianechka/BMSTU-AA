\chapter{Технологическая часть}
В этом разделе будут приведены листинги кода и результаты функционального тестирования.

\section{Средства реализации программного обеспечения}
В качестве языка программирования выбран Python 3.9, так как имеется опыт разработки проектов на этом языке.
Для замера процессорного времени используется функция process\_time\_ns из библиотеки time. 
В её результат не включается время, когда процессор не выполняет задачу \cite{python}.

\section{Листинг кода}
\FloatBarrier
На листинге 3.1 представлена реализация алгоритма полного перебора.
На листинге 3.2 представлена реализация муравьиного алгоритма. 

\begin{lstinputlisting}[language=Python, caption=Реализация алгоритма полного перебора, linerange={5-23}, 
	basicstyle=\footnotesize\ttfamily, frame=single,breaklines=true]{src/main.py}
\end{lstinputlisting}
\FloatBarrier

\FloatBarrier
\begin{lstinputlisting}[language=Python, caption=Реализация муравьиного алгоритма, linerange={24-121}, 
	basicstyle=\footnotesize\ttfamily, frame=single, breaklines=true]{src/main.py}
\end{lstinputlisting}
\FloatBarrier

\section{Функциональные тесты}
В таблице 3.1 приведены результаты функциональных тестов. 
В первом столбце -- исходная матрица расстояний.
Во втором столбце -- ожидаемый результат: кратчайший путь
В третьем столбце -- его длина.

\FloatBarrier
\begin{table}[h]
	\caption{Функциональные тесты}
	\begin{tabular}{| c | c | c |}
		\hline
		Матрица & Кратчайший путь & Длина \\ \hline
		\begin{tabular}{c c} 
			0 & 1 \\
			1 & 0
		\end{tabular}
		& [0, 1, 0] & 2\\
		\hline
		
		\begin{tabular}{c c c c c} 
			0 & 5 & 2 & 3 & 4 \\
			5 & 0 & 7 & 1 & 2 \\
			2 & 7 & 0 & 9 & 5 \\ 
			3 & 1 & 9 & 0 & 2 \\
			4 & 2 & 5 & 2 & 0 \\
		\end{tabular}
		&
		[0, 2, 4, 1, 3, 0] & 13 \\
		\hline
		
		\begin{tabular}{c c c c c} 
			0 & 7 & 2 & 9 & 4 \\
			2 & 0 & 7 & 1 & 3 \\
			2 & 7 & 0 & 9 & 5 \\
			1 & 1 & 2 & 0 & 2 \\
			4 & 7 & 5 & 2 & 0 \\
		\end{tabular}
		&
		[0, 2, 4, 3, 1, 0] & 12 \\
		\hline
		
		\begin{tabular}{c c c c c} 
			0 & 1 & 1 & 1 & 1  \\
			1 & 0 & 1 & 1 & 1 \\
			1 & 1 & 0 & 1 & 1 \\
			1 & 1 & 1 & 0 & 1 \\
			1 & 1 & 1 & 1 & 0 \\
		\end{tabular}
		& 
		[0, 1, 2, 3, 4, 0] & 5 \\
		\hline
	\end{tabular}
\end{table}
\FloatBarrier
Все тесты алгоритмами были пройдены успешно.

\section{Вывод}
Были приведены листинги кода.
Были проведены функциональные тесты.
Все алгоритмы справились с тестированием.