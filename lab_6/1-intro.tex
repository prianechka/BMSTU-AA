\newpage
\addcontentsline{toc}{chapter}{Введение}

\chapter*{Введение}
В современной теории алгоритмов есть задачи, которые возможно решить только полным перебором всех вариантов.
Но для многих задач невозможно вычислить решение за полиномиальное время.
Поэтому существуют алгоритмы, которые позволяют найти решение, близкое к идеальному, за достижимое время.

Муравьиный алгоритм — один из эффективных полиномиальных алгоритмов для нахождения приближенных решений задачи коммивояжёра, 
а также решения аналогичных задач поиска маршрутов на графах.

Цель лабораторной работы -- разработать ПО, которое решает задачу коммивояжёра двумя способами: полным перебором и муравьиным алгоритмом.

Для достижения поставленной цели необходимо выполнить следующее:
\begin{itemize}
	\item рассмотреть формализацию задачи коммивояжёра;
	\item рассмотреть муравьиный алгоритм;
	\item привести схемы реализации алгоритмов;
	\item определить средства программной реализации;
	\item реализовать два алгоритма для решения задачи;
	\item протестировать разработанное ПО;
	\item оценить реализацию алгоритмов по времени и памяти.
\end{itemize}