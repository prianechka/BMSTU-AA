\chapter{Заключение}
В ходе работы была рассмотрена задача коммивояжёра и два алгоритма для её решения: муравьиный и полный перебор.
Были составлены схемы алгоритмов.
Было реализовано работоспособное ПО, удалось провести анализ зависимости затрат по времени от размера матрицы расстояний.

Применимость алгоритмов зависит от того, насколько велик размер матрицы расстояний.
При размерах $ N < 9 $ алгоритм полного перебора работает быстрее, чем муравьиный алгоритм.
При $ N = 2 $ перебор быстрее в 275 раз, а при $ N = 8 $ в 4 раза.

Но при дальнейшем увеличении матрицы муравьиный алгоритм работает быстрее.
При $ N = 12 $ муравьиный алгоритм превосходит перебор в 82 раза.
При $ N > 12 $ время для подбора подсчитать не удалось -- оно слишком велико.

Муравьиный алгоритм при различных параметрах выдаёт разные ответы.
Регулировка параметров производится вручную, и подходящие значения параметров зависят от задачи.