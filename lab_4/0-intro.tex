\newpage
\addcontentsline{toc}{chapter}{Введение}

\chapter*{Введение}
В машинном обучении и аналитике данных часто встаёт задача нахождения взаимосвязи между двумя переменными.\cite{ml}
Для этого используется такая метрика как корреляция Пирсона.

Актуальность работы заключается в том, нахождение этого значения тратит много системных ресурсов, в первую очередь времени.
Для сокращения временных затрат можно применить параллельные вычисления.

Целью данной работы является разработка программы, которая реализует два способа нахождения двух самых коррелирующих
признаков в признаковом пространстве: классический, и с использованием потоков.

Для достижения поставленной цели необходимо выполнить следующее:
\begin{itemize}
	\item рассмотреть алгоритм нахождения корреляции Пирсона;
	\item рассмотреть способы распараллеливания алгоритма;
	\item привести схемы реализации рассматриваемых алгоритмов;
	\item определить средства программной реализации;
	\item реализовать рассматриваемые алгоритмы;
	\item протестировать разработанное ПО;
	\item провести модульное тестирование всех реализаций алгоритмов;
	\item оценить реализацию алгоритмов по времени и памяти.
\end{itemize}