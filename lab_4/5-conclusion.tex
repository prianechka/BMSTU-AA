\chapter{Заключение}
В ходе работы были рассмотрены два алгоритма поиска наиболее коррелирующих столбцов: классический алгоритм
и многопоточный.
Были составлены схемы алгоритмов.
Было реализовано работоспособное ПО, удалось провести анализ зависимости затрат по времени от размера матрицы и количества потоков.

Многопоточный алгоритм показал более высокие результаты, начиная с $N=50$.
16-поточный алгоритм показал наилучшие результаты на матрицах размером $N >= 200$. 
Классический алгоритм на небольших размерах матрицы оказался быстрее, не требуя при этом
поддержки поточности.
Количество потоков, большее, чем количество ядер, не дало прироста до $N = 500$, а затем
увеличило результаты на 20\%.