\chapter{Технологический раздел}
В данном разделе приведены требования к ПО, выбранные средства реализации и листинги кода.
\section{Средства реализации программного обеспечения}
В качестве языка программирования выбран Python 3.9, так как имеется опыт разработки проектов на этом языке.
Для замера процессорного времени используется функция process\_time из библиотеки time. В её результат не включается время, когда процессор не выполняет задачу \cite{python}. 

\section{Требования к ПО}
Ниже будет представлен список требований к разрабатываемому программному обеспечению. 

Требования к входным данным: 
\begin{itemize}
	\item вводимые строки должны быть регистрозависимыми;
	\item язык строк должен быть русский или английский;
	\item строки могут быть пустыми -- это не ошибочная ситуация.
\end{itemize}

Требования к выводу: 
\begin{itemize}
	\item программа должна выводить числовое значение (расстояние) для введенных строк по каждому алгоритму;
	\item программа должна выводить матрицу расстояний, если она использовалась в алгоритме.
\end{itemize} 

\section{Листинг кода}
\begin{lstinputlisting}[language=Python, caption=Рекурсивная реализация алгоритма поиска расстояния Левенштейна, linerange={3-16}, basicstyle=\small\sffamily, frame=single]{src/lab_1.py}
\end{lstinputlisting}

\begin{lstinputlisting}[language=Python, caption=Рекурсивная реализация (с матрицей) алгоритма поиска расстояния Левенштейна, linerange={17-38}, basicstyle=\small\sffamily, frame=single]{src/lab_1.py}
\end{lstinputlisting}

\begin{lstinputlisting}[language=Python, caption=Матричная реализация (с хранением двух строк) алгоритма поиска расстояния Левенштейна, linerange={49-66}, basicstyle=\small\sffamily, frame=single]{src/lab_1.py}
\end{lstinputlisting}

\begin{lstinputlisting}[language=Python, caption=Рекурсивная реализация (с матрицей) алгоритма поиска расстояния Дамерау-Левенштейна, linerange={67-98}, basicstyle=\small\sffamily, frame=single]{src/lab_1.py}
\end{lstinputlisting}


\clearpage
\section{Функциональные тесты}
В таблице 3.1 приведены результаты функциональных тестов. Первые три числа в столбцах с ответами - результаты работы алгоритмов для нахождения расстояния Левенштейна, последний столбец - результат работы алгоритма нахождения расстояния Дамерау-Левенштайна

\begin{table}[!ht]
  \caption{Результаты функциональных тестов}
  \centering
\begin{tabular}{ | l | l | l | l | }
\hline
Строка 1 & Строка 2 & Ожидаемый результат & Фактический результат \\ \hline
скат & кот & 2 2 2 2 & 2 2 2 2 \\
увлечения & развлечения & 4 4 4 4 & 4 4 4 4\\
АААААА & ББББББ & 6 6 6 6& 6 6 6 6\\
хотдог & каток & 4 4 4 4 & 4 4 4 4\\
 &  & 0 0 0 0 & 0 0 0 0\\
общага & общага & 0 0 0 0 & 0 0 0 0\\
привет & Пока & 6 6 6 6 & 6 6 6 6\\
привет & пока & 5 5 5 5 & 5 5 5 5 \\
привет & првиет & 2 2 2 1 & 2 2 2 1\\
*oбщага & общага & 1 1 1 1 & 1 1 1 1\\
1234 & 2143 & 3 3 3 2 & 3 3 3 2\\
\hline
\end{tabular}
\end{table}

* - первая буква в первой строке - на английской раскладке.

Все тесты пройдены успешно.

\section{Вывод}

Были разработаны и протестированы спроектированные алгоритмы: вычисления расстояния Левенштейна рекурсивно, с заполнением матрицы и рекурсивно с заполнением матрицы, а также вычисления расстояния Дамерау — Левенштейна с заполнением матрицы.
