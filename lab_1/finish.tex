\chapter{Заключение}

В ходе выполнения работы были выполнены все поставленные задачи и изучены методы динамического программирования на основе алгоритмов вычисления расстояния Левенштейна.

Экспериментально были установлены различия в производительности различных алгоритмов вычисления расстояния Левенштейна. Для слов длины 10 рекурсивный алгоритм Левенштейна работает на несколько порядков медленнее (160000 раз) матричной реализации. Рекурсивный алгоритм с параллельным заполнением матрицы работает быстрее простого рекурсивного (в 29000 раз), но все еще медленнее матричного (в 5.5 раз). Если длина сравниваемых строк превышает 10, рекурсивный алгоритм становится неприемлимым для использования по времени выполнения программы. Матричная реализация алгоритма Дамерау — Левенштейна сопоставимо с алгоритмом Левенштейна. В ней добавлены дополнительные проверки, но, эти алгоритмы находятся в разных полях использования.

Теоретически было рассчитано использования памяти в каждом из алгоритмов вычисления расстояния Левенштейна. Обычные матричные алгоритмы потребляют намного больше памяти, чем рекурсивные, за счет дополнительного выделения памяти под матрицу. Оптимизация матричного варианта позволяет сократить объём потребляемой памяти, что делает его самым производительным вариантом.