\newpage
\addcontentsline{toc}{chapter}{Введение}

\chapter*{Введение}
Расстояние Левенштейна\cite{Levenshtein} - мера, которая определяет, насколько отличаются друг от друга две строки. Фактически величина показывает, сколько нужно произвести односимвольных изменений, чтобы преобразовать одно слово к другому.

Расстояние Левенштейна получило широкое применение в компьютерной лингвистике. Активно его использует Яндекс\cite{yandex} для поисковых систем. Мера используется для:
\begin{itemize}
    \item Автоматического исправления ошибок в тексте
    \item Поиска возможных ошибок в поисковых запросах
    \item Расчёта изменений в различных версиях текста (утилита diff)
\end{itemize}

Также расстояние Левенштейна нашло применение в биоинформатике для нахождения разности последовательностей генов.

Актуальность работы заключается в том, что нахождение расстояния Левенштейна должно выполняться за максимально короткое время.
Цель работы - разработать программу, реализующую различные способы нахождения расстояния Левенштейна. 
Поставленные задачи:
\begin{itemize}
\item Изучение алгоритмов нахождения расстояния Левенштейна и Дамерау-Левенштейна
\item Реализация рекусирсионных и итеративного алгоритмов нахождения расстояния Левенштейна
\item Сравнительный анализ по времени и памяти алгоритмов нахождения расстояния Левенштейна
\item Реализация алгоритма нахождения расстояния Дамерау-Левенштейна
\end{itemize}