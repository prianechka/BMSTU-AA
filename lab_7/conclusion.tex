\chapter{Заключение}
В ходе работы было рассмотрено понятие словаря и рассмотрены три алгоритма поиска в словаре: 
полный перебор, бинарный поиск и поиск с использованием сегментов.
Были составлены схемы алгоритмов.
Было реализовано работоспособное ПО, удалось провести анализ зависимости затрат по времени от алгоритма.

Применимость алгоритмов зависит от того, насколько велик размер словаря.
Алгоритм полного перебора перестаёт оправдывать себя после 1000-го ключа, а на 20000-м ключе
он уступает по времени бинарному поиску в 7 раз, а поиску с использованием сегментов -- в 70 раз.

Алгоритм бинарного поиска по времени оказался самым медленным при номере ключа до 1000.
Алгоритм поиска с использованием сегментов оказался самым быстрым во всех случаях.
При этом оба алгоритма работают за примерно одинаковое время при любом номере ключа.
Количество сравнений для этих алгоритмов не больше 20.