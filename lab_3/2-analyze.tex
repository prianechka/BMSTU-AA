\chapter{Аналитический раздел}
В данном разделе будут рассмотрены три алгоритма сортировки.
Будут проанализированы их преимущества и недостатки.
Будет произведена общая формализация задачи.

\section{Формализация задачи}
Пусть A - массив, содержащий N чисел.
Массив считается отсортированным по возрастанию, если для каждого элемента массива выполняется следующее соотношение \ref{sort}:
\begin{equation}
	\label{sort}
		\forall i, j   \in [1, N] : i  \leq  j ; A_i \leq A_j 
\end{equation}

Массив считается отсортированным по убыванию, если для каждого элемента массива выполняется следующее соотношение \ref{back-sort}:
\begin{equation}
	\label{back-sort}
		\forall i, j   \in [1, N] : i  \geq  j ; A_i \geq A_j 
\end{equation}

Для выполнения цели работы требуется написать алгоритм, который для произвольных данных сортирует по возрастанию массив.

\section{Сортировка пузырьком}
Алгоритм состоит из повторяющихся проходов по сортируемому массиву \cite{bubble}. 
За каждый проход элементы последовательно сравниваются попарно и, если порядок в паре неверный, выполняется перестановка элементов. 
Проходы по массиву повторяются N-1 раз или до тех пор, пока на очередном проходе не окажется, что обмены больше не нужны, что означает — массив отсортирован. 
При каждом проходе алгоритма по внутреннему циклу, очередной наибольший элемент массива ставится на своё место в конце массива рядом с предыдущим «наибольшим элементом», а наименьший элемент перемещается на одну позицию к началу массива, если ещё не находится там.

Недостаток алгоритма состоит в том, что наименьшие элементы, находящиеся в конце массива, будут за каждую итерацию смещаться к правильному положению лишь на одну позицию.
Алгоритм эффективен лишь на небольших массивах, для практических задач не используется.

К преимуществам алгоритма стоит отнести простоту реализации и отсутствие дополнительных затрат на память.

\section{Сортировка вставками}
Сортировка вставками - алгоритм сортировки, в котором элементы входной последовательности рассматриваются по одному, 
и каждый новый поступивший элемент размещается в подходящем месте среди ранее упорядоченных элементов\cite{insert}.

В начальный момент последовательность отсортированных элементов пуста. 
На каждом шаге алгоритма выбирается один из элементов входных данных и помещается на нужную позицию до тех пор,
пока набор данных не будет исчерпан.

Для того, чтобы оптимизировать затраты на память, отсортированная последовательность будет храниться в начале массива.
То есть на каждой итерации элемент ставится на правильное место относительно тех элементов, которые рассматривались 
на предыдущих итерациях.

К преимуществам можно отности то, что в случае удачного расположения элементов алгоритм будет работать за линейное время.

\section{Сортировка выбором}
Сортировка выбором -- сортировка, на каждой итерации которого алгоритм ищет минимальный элемент в массиве
и располагает его сразу на правильное место в отсортированном массиве. \cite{select}

Алгоритм можно разбить на три шага:
\begin{enumerate}
	\item Найти индекс минимального элемента в массиве.
	\item Произвести обмен минимального элемента в массиве с первым неотсортированным по известным индексам.
	\item Проделать те же операции для ещё не отсортированных элементов в массиве.
\end{enumerate}

Чтобы обеспечить устойчивость алгоритма, необходимо минимальный элемент непосредствнно вставлять в первую неотсортированную
позицию, не меняя порядок остальных элементов.

\section{Вывод}
Были рассмотрены три алгоритма сортировок, проведена формализация задачи,
проанализированы преимущества и недостатки алгоритмов.

