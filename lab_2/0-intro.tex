\newpage
\addcontentsline{toc}{chapter}{Введение}

\chapter*{Введение}
Матричные вычисления - основа многих алгоритмов. Матрицы постоянно используются в компьютерной графике,
моделировании физических экспериментов, реализации цифровых фильтров. 
В алгоритмах по построению рекомендательных систем также используется умножение матриц. 

Актуальность работы заключается в том, что умножение матриц - затратная операция, поэтому требуется
подбирать алгоритмы, которые позволяют оптимизировать эти действия.

Целью данной работы является разработка программы, которая реализует три алгоритма умножения матриц:
стандартный алгоритм, алгоритм Винограда и модифицированный алгоритм Винограда.

Для достижения поставленной цели необходимо выполнить следующее:
\begin{itemize}
	\item рассмотреть существующие алгоритмы умножения матриц;
	\item привести схемы реализации рассматриваемых алгоритмов;
	\item определить средства программной реализации;
	\item реализовать рассматриваемые алгоритмы;
	\item протестировать разработанное ПО;
	\item провести модульное тестирование всех реализаций алгоритмов;
	\item оценить реализацию алгоритмов по времени и памяти.
\end{itemize}