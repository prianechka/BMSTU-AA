\newpage
\addcontentsline{toc}{chapter}{Введение}

\chapter*{Введение}
Многие задачи требуют огромных вычислительных мощностей, но даже суперкомпьютеры не всегда справляются с поставленными задачами.
Вычислительные конвейеры предоставляют возможность распараллелить исходный алгоритм для оптимального использования системных ресурсов.
Вычислительные конвейеры используются в процессорах для ускорения выполнения инструкций.

Целью данной работы является разработка программы, которая реализует два способа реализации
вычислительного конвейера: линейный и с использованием параллельных вычислений.

Для достижения поставленной цели необходимо выполнить следующее:
\begin{itemize}
	\item рассмотреть методологию конвейерной разработки;
	\item рассмотреть способы распараллеливания конвейерных вычислений;
	\item привести схемы реализации конвейера;
	\item определить средства программной реализации;
	\item реализовать рассматриваемые способы реализации вычислительного конвейера;
	\item протестировать разработанное ПО;
	\item провести модульное тестирование всех реализаций алгоритмов;
	\item оценить реализацию алгоритмов по времени и памяти.
\end{itemize}