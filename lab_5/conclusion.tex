\chapter{Заключение}
В ходе работы была рассмотрена методология конвейерных вычислений на примере задачи стандартизации данных.
Были составлены схемы алгоритмов.
Было реализовано работоспособное ПО, удалось провести анализ зависимости затрат по времени от размера массива и количества массивов.

Вычислительный конвейер оказался быстрее на всех тестах.
При количестве массивов $ N = 10 $ параллельный алгоритм работает в 2.06 раз быстрее, чем линейная реализация.
При количестве массивов $ N = 1000 $ разница достигла 2.3 раз.
Уже при размере очереди в два элемента реализация вычислительного конвейера оправдывает себя.

При фиксированном количестве массивов (500) параллельный конвейер также быстрее при любом размере массива.
При размере массива $ N = 50 $ параллельная реализация опередила линейную в 5.34 раза. 
При $ N = 5000 $ конвейер был быстрее в 1.6 раз, при $ N = 10000 $ --  в 2.3 раза.

Конвейерные вычисления позволили существенно увеличить скорость выполнения задачи.